\section{Introduction}
\label{sec:introduction}

With the prevalence of cloud-native software systems, 
    Kubernetes has become one of the popular choices for orchestrating productionized container applications. 
Developers extend Kubernetes APIs and functionalities by writing specialized programs called operators, which automate the laborious task of deploying and managing services and applications. 
Since operators are responsible for various highly complex operations such as service scaling, backup, and migration in production environment, their correctness is paramount for system reliability. 
To understand the reliability of operators and how it could be improved, we must first develop a deep understanding of the complexity of operators and operators fail.

The significance of reliability of Kubernetes operators stems from challenges inherent in their design and execution within production environments. 
Operators undertake diverse and mission-critical responsibilities, including but not limited to dynamic scaling of services, data backup, and seamless application migration. 
As a result, any failures within operator functionalities can cause cascading repercussions, ranging from service disruptions to potential data loss, ultimately undermining the stability and availability of the entire system.
