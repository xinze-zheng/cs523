\section{Methods}
To understand the reliability implications in Kubernetes operators, we collect
various metrics from 52 popular open-sourced operators. Each of the operators
is then analyzed to understand the complexity of the operators and the
challenges faced by the developers in writing reliable operators.

\subsection{Operator Selection}
The 52 operators are selected by students in the UIUC Spring 2024 CS523
Advanced Operating Systems course. Each student is instructed to choose one
operators that they are interested in and to ensure the operators are popular,
mature, and well-maintained. Popularity and maturity are measured by the number
of stars and forks on GitHub, and the maintenance status is measured by the last
commit date.

The operators are selected from a variety of domains, including but not limited
to message queue, database system, and distributed synchronization systems.

The operators selection might not be representative of the entire Kubernetes
operator ecosystem. The selection might be biased towards the students' interest
and the managed system they are familiar with.

\subsection{Data Collection}
We collect the following metrics from the operators:

\begin{itemize}
  \item \textbf{LOC and programming language distribution}: We collect the
        number of lines of code (LOC) and the programming language used in
        writing the operators. We use the LOC as a proxy for the complexity of
        the operators.
  \item \textbf{Operator developers' involvement in the managed system
          development}: We collect the involvement of operator developers in
        contributing to the managed system using GitHub repository contributor
        data. We use this metric to understand the familiarity of operator
        developers with the managed system.
  % \item \textbf{Managed system complexity}: We collect the number of APIs
  %     used by the operators to interact with the managed system. We use this
  %     metric to understand the complexity of the managed system.
\end{itemize}
